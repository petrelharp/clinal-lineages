\documentclass{article}

\usepackage{fullpage}
\usepackage{amsmath}
\usepackage{amssymb}

\newcommand{\E}{\mathbb{E}}
\renewcommand{\P}{\mathbb{P}}
\newcommand{\deriv}[1]{\frac{d}{d#1}}
\newcommand{\dderiv}[1]{\frac{d^2}{d^2#1}}
\newcommand{\given}{\;\vert\;}

\begin{document}

\section{Discrete model}

Consider a discrete model in which the total number of individuals at location $x$ is $N(x)$,
each individual is of type either B or b at a selected locus,
and the number of individuals of type B and location $x$ and time $t$ is $n(x,t)$
(but, we often neglect the $t$).
Suppose that type B individuals at location $x$ reproduce at rate $s_B(x)$, 
and likewise type b at rate $s_b(x)$.
At reproduction, individuals recombine with others in the same location,
and the offspring choose a new location according to the kernel $m(x,y)$.
The population dynamics are random, but suppose that $N(x)$ is sufficiently large that these do not vary with time;
instead we want to follow a lineage backwards in time,
at a locus separated from the selected locus by recombination fraction $r$.

Suppose this is a Moran model.
There are four things that can happen:
\begin{enumerate}
    \item[$x\xrightarrow{BB}y$] One type B individual at location $x$ reproduces, 
        either does not recombine or recombines with another type B,
        and sends the offspring to $y$.
    \item[$x\xrightarrow{Bb}y$] An individual at location $x$ reproduces, 
        recombines with the other type,
        and sends to $y$ an offspring
        who inherits at the selected locus from the type b parent 
        and the at the neutral locus from the type B parent.
    \item[$x\xrightarrow{bB}y$] An individual at location $x$ reproduces, 
        recombines with the other type,
        and sends to $y$ an offspring
        who inherits at the selected locus from the type B parent 
        and the at the neutral locus from the type b parent.
    \item[$x\xrightarrow{bb}y$] One type b individual at location $x$ reproduces, 
        either does not recombine or recombines with another type b,
        and sends the offspring to $y$.
\end{enumerate}
Let $p(x) = n(x)/N(x)$.
These four things happen at rates:
\begin{align}
    & x\xrightarrow{BB}y & \qquad w_{BB}(x,y) &= p(x) s_B(x) \left(1 - r (1-p(x)) \right)  N(x) m(x,y) \\
    & x\xrightarrow{Bb}y & \qquad w_{Bb}(x,y) &= r p(x) (1-p(x)) \frac{s_B(x)+s_b(x)}{2} N(x) m(x,y) \\
    & x\xrightarrow{bB}y & \qquad w_{bB}(x,y) &= r p(x) (1-p(x)) \frac{s_B(x)+s_b(x)}{2} N(x) m(x,y) \\
    & x\xrightarrow{bb}y & \qquad w_{bb}(x,y) &= (1-p(x)) s_b(x) \left(1 - r p(x) \right) N(x) m(x,y) 
\end{align}

Note that at equilibrium, we require
\begin{align}
  0 = \sum_y N(y) m(y,x) \left\{ p(y)s_B(y)(1-p(x)) - (1-p(y))s_b(y)p(x) \right\} .
\end{align}

\subsection{Lineage movement}

These rates tell us the rates at which a lineage will move, backwards in time.
For instance, the rate at which a lineage at the selected locus
currently in a type B individual at location $x$
jumps to another type B individual at location $y$ is equal to the rate of influx of migrants from $y$
divided by the number of B alleles at $x$,
or
\begin{align}
  r_A(x,y) &= \frac{w_{BB}(y,x)}{p(x)N(x)} \\
  &= N(y) p(y)  s_B(y) m(y,x) \frac{ 1 }{ N(x) p(x) }
\end{align}

If we follow a linked locus until the first time it is found in a type b individual,
it moves according to these jump rates, and at location $x$ recombines at rate
\begin{align}
  \gamma_B(x) &= \frac{1}{N(x)p(x)}\sum_y w_{bB}(y,x)  \\
  &= r p(y) (1-p(y)) \frac{s_B(y)+s_b(y)}{2} N(y) m(y,x) \frac{1}{N(x)p(x)}.
\end{align}
Suppose that we kill the lineage when it recombines onto type b.
Let $X_t$ denote the position of the lineage at time $t$ in the past,
with $X_t = \rho$ if it has recombined onto b,
and let $f$ be test function with $f(\rho)=0$.
If we let $M_r(y) = s_B(y) \left(1 - r (1-p(y)) \right)$, then
\begin{align}
    \deriv{t} \E[f(X_t) \given X_0=x ] &= \sum_y r_{B}(y,x) ( f(y)-f(x) ) - \gamma_B(x) f(x) \\
    &= \frac{1}{N(x)p(x)} \sum_y  N(y) p(y) M_r(y) m(y,x) ( f(y) - f(x)  ) - \gamma_B(x) f(x) . \label{eqn:discrete_generator}
\end{align}

\section{Diffusion limit}

Now suppose that $m(x,y)$ is symmetric, and depends on a parameter $\sigma$ so that as $\sigma \to 0$,
the associate random walk converges to Brownian motion, so that
\begin{align}
    \lim_{\sigma \to 0} \sum_y \frac{ m(x,y) ( f(y) - f(x) ) }{\sigma^2} = \frac{1}{2} \dderiv{x} f(x) .
\end{align}
Write $f'(x) = \deriv{x}f(x)$, and note that
\begin{align}
    \frac{1}{\sigma^2} \sum_y g(y) m(y,x) (f(y)-f(x)) 
    &= \frac{1}{\sigma^2} \sum_y m(y,x) \left( g(y) f(y) - g(x) f(x) + (g(x)-g(y)) f(x) \right) \\
    &= \frac{1}{\sigma^2} \sum_y m(y,x) \left( g(y) f(y) - g(x) f(x) \right) \\
    & \qquad - f(x) \frac{1}{\sigma^2} \sum_y m(y,x) (g(y)-g(x)) \\
    &\xrightarrow{\sigma \to 0} \frac{1}{2} \dderiv{x}\left( g(x)f(x) \right) - \frac{1}{2} f(x) \dderiv{x} g(x) \\
    &= \frac{1}{2} \left( g(x) f'(x) + 2 g'(x) f'(x) + f(x) g''(x) - f(x) g''(x) \right) \\
    &= \frac{1}{2} g(x) f''(x) + g'(x) f'(x) . \label{eqn:deriv_limit}
\end{align}

If we also rescale recombination, so that $\rho = r/\sigma^2$,
then 
\begin{align}
    \frac{1}{\sigma^2} \gamma_B(x) \xrightarrow{\sigma \to 0} \rho (1-p(x)) \frac{s_B(x)+s_b(x)}{2} ,
\end{align}
which we can define to be equal to $\rho k(x)$.
In this case, note that since $r = \rho \sigma^2$,
\begin{align}
    M_r(x) \xrightarrow{\sigma \to 0} s_B(x) .
\end{align}
Under these assumptions, combining \eqref{eqn:discrete_generator} and \eqref{eqn:deriv_limit},
\begin{align}
    \deriv{t} \E[f(X_{t/\sigma^2}) \given X_0=x ] &\xrightarrow{\sigma \to 0} 
    \frac{1}{2} s_B(x) \dderiv{x} f(x) + \frac{1}{N(x)p(x)} \deriv{x} \left\{ N(x)p(x) s_B(x) \right\}  \deriv{x} f(x) - \rho k(x) f(x) \\
    &= \frac{1}{2} s_B(x) f''(x) + \left( s_B'(x) + \log(N(x)p(x))' s_B(x) \right) f'(x)  - \rho k(x) f(x) ,
\end{align}
i.e.\ $X_{t/\sigma^2}$ converges to a diffusion with drift $\frac{1}{ N(x)p(x)} \deriv{x} ( N(x) p(x) s_B(x) )$ and killed at rate $\rho k(x)$.

\paragraph{Stationary distribution}
If $X$ has a stationary distribution then its density $\pi(x)$ solves the forward Kolmogorov equation,
which is here (letting $n(x) = N(x) p(x)$ for convenience)
\begin{align}
  0 &= \frac{1}{2} \dderiv{x} \left( s_B(x) \pi(x) \right) - \deriv{x} \left( \frac{\pi(x)}{n(x)} \deriv{x}(n(x) s_B(x)) \right) .
\end{align}
It is easy to check that $\pi(x) \propto n(x)^2 s(x) $ is a solution,
and since it is nonnegative, the stationary distribution,
i.e.
\begin{align}
  \pi(x) = \frac{ \left( N(x) p(x) \right)^2 s_B(x) }{ \int_y \left( N(y) p(y) \right)^2 s_B(y) dy } ,
\end{align}
if the denominator (below denoted by $Z$) is finite.

In other words, the stationary distribution is proportional to the local rate of production of $B$ alleles, \emph{squared}.


\paragraph{Speed and scale}
It will be useful to think about the Dubins-Schwartz construction of this diffusion.
Note that if we define
\begin{align}
  z(x) = \int_{0}^x \frac{ dy }{ n(y)^2 s_B(y)^2 },
\end{align}
where $n(x) = N(x) p(x)$ as before,
then
\begin{align}
  \deriv{t} \E^x[z(X_t)] &= \frac{1}{2} s_B(x) \deriv{x} \frac{1}{ n(x)^2 s_B(x)^2 } + \frac{1}{n(x)} \frac{ \deriv{x}(n(x)s_B(x)) }{  n(x)^2 s_B(x)^2 } \\
  &= - s_B(x) \frac{ \deriv{x} (n(x)s_B(x)) }{ n(x)^3 s_B(x)^3 } + \frac{1}{n(x)} \frac{ \deriv{x}(n(x)s_B(x)) }{  n(x)^2 s_B(x)^2 } \\
  &= 0 ,
\end{align}
that $Z_t = z(X_t)$ is a martingale.

Since
\begin{align}
  dX_t = \frac{\deriv{x}(n(X_t)s_B(X_t))}{n(X_t)} dt + \sqrt{s_B(X_t)} dB_t,
\end{align}
then if $s_B(x)$ is bounded below, the lineage won't get stuck,
and will always come back to the patch.
Since $z'(x) = 1/(n(x)s_B(x))^2$,
applying It\^o's formula,
\begin{align}
  dZ_t &=  z'(X_t) dX_t + \frac{1}{2} z''(X_t) s_B(X_t) dt \\
  &= \frac{ dB_t }{ n^2(X_t) s_B^{3/2}(X_t) } ,
\end{align}
i.e.\ $Z_t$ ``looks like Brownian motion moving at speed $1/(n^4(X_t) s_B^3(X_t))$''.
This happens because the transformation $z(x)$ squooshes up locations far away from the patch,
where the lineage moves at a more-or-less constant speed;
so that $Z_t$ has to move faster in those squooshed-up regions.


\paragraph{Rescaling selection out}
Suppose for the moment that $s_b(x)=1$. 
It hasn't entered here much anyhow.
First note that if we change spatial variables to
\begin{align}
  s(x) = \int_0^x \sqrt{s_B(y)} dy
\end{align}
then $ds = \sqrt{s(x)} dx$, and the generator for the process becomes
\begin{align}
  f \mapsto 
    \frac{1}{2} f''(s) + \log\left( N(s) p(s) s_B(s) \right)' f'(s) - \rho k(s) f(s) .
\end{align}
At first sight this looks like the usual scaling of $\sigma/\sqrt{s}$,
but it isn't, because in the notation I'm using there, $s_B$ is around 1,
i.e.\ $s_B = 1+s$ if $s$ is the usual selection coefficient.


\paragraph{Recombination time}
Let $\tau_\rho$ be the first time  the lineage recombines onto type b.
If $\rho$ is much smaller than the mixing time of a type B lineage,
then the time to recombination is roughly Exponential,
with mean
\begin{align}
    \E[\tau_\rho] \approx \frac{1}{ \rho \int \pi(x) k(x) dx } .
\end{align}
The product $\pi(x) k(x)$ is:
\begin{align}
    \pi(x) k(x) = \frac{1}{Z} N(x)^2 p(x)^2 s_B(x) (1-p(x)) (s_B(x)+s_b(x))/2 .
\end{align}

More generally, if we define
\begin{align}
    h(x) = \E[\tau_\rho \given X_0 = x]
\end{align}
then $h$ solves the equation
\begin{align}
  -1 = \frac{1}{n(x)}(n(x)s_B(x))' h'(x) + \frac{1}{2} s_B(x) h''(x) - \rho k(x) h(x) .
\end{align}

The probability that a segment of length $r = \rho \sigma^2$ to one side of the selected locus
does not recombine across time $t$ is $\P\{\tau_\rho > t\}$.
Therefore, if $L_t$ is the length of the nonrecombined segment to one side of the selected locus
after time $t$, then
\begin{align}
    \E[ L_t \given X_0=x ] &= \sigma^2 \int_0^\infty \P\{ \tau_\rho > t \} d\rho \\
\end{align}

The appropriate thing to do is to look at $\E[L_T]$, where $T$ is the coalescence time.
Generally, then
\begin{align}
  \P\{ L_t > \rho \} = \P\{ \tau_\rho > T \} .
\end{align}
Do do this properly, we need to run two diffusions.
However, if we let $T$ instead be independent Exponential($\lambda$), then 
\begin{align}
  \ell(x) = \P\{ \tau_\rho > T \given X_0 = x\} 
\end{align}
solves the equation
\begin{align}
  \frac{1}{n(x)}(n(x)s_B(x))' h'(x) + \frac{1}{2} s_B(x) h''(x) - (\lambda + \rho k(x)) h(x) = \lambda .
\end{align}


%%%%%%%%%
\section{Integrals against $p$}

Recall that if $N(x)$ is constant, the equilibrium frequency $p(x)$ solves
\[ \frac{1}{2} p''(x) + s(x) p(x)(1-p(x)) = 0 \quad ,\]
where $s(x) = s_B(x)-s_b(x)$,
and that if $s(x) = s$ is constant on a region, the \emph{inverse} solves
\begin{align}
  \frac{dx}{dp} = \frac{1}{\sqrt{sp^2(2p/3-1) + K}},
\end{align}
where $K$ is a constant to make the boundary conditions come out right.

If this is the case, then we can change variables:
\begin{align}
  \int_a^b f(p(x)) dx = \int_{p(a)}^{p(b)} \frac{f(p) dp }{\sqrt{sp^2(2p/3-1) + K}} .
\end{align}
so for instance, 
the denominator for the stationary distribution (dropping the $N$'s)
\begin{align}
  Z &= \int p(y)^2 s_B(y) dy \\
  &= \int \frac{ p (1+s) dp }{ \sqrt{ s (2p/3-1) + K } } .
\end{align}
Note, however, that the first expression makes it clear that if we make the patch bigger, that $Z$ will increase roughly proportionally
-- since this increases the area over which $p(y)\approx 1$.
To make this occur in the integral over $p$ on the second line, 
it must be the case that $K$ is approximately $s/3$ plus a term inversely proportional to the area of the patch.


The integral of $\pi(x) k(x)$ is
\begin{align}
  \frac{1}{Z} \int_a^b p(x)^2(1-p(x))s_B(x)(s_b(x)+s_B(x)) dx = \frac{1}{Z} \int_{p(a)}^{p(b)} \frac{p (1-p) (1+s)(1+s/2) dp }{\sqrt{s(2p/3-1) + K}} .
\end{align}
I have the feeling Barton does similar things in one of his papers,
but haven't been able to track it down.


%%%%%%
\section{Haplotype lengths}

Some statements above about ``haplotype length'' aren't quite right.
Consider the haplotype shared about two B alleles;
by the time these coalesce, their ``shared haplotype'' in the above sense may be broken up into a number of blocks --
all coalescing linked to other B alleles.
More properly, the length of the contiguous haplotype
is determined by branching diffusions.
The single-lineage version is thus:
begin a diffusion with mass equal to genetic length (of something, rescaled);
then, any remaining particles split in two at rate proportional to their length,
always labeling the particle corresponding to the selected allele;
when this happens any particles other than the labeled particle may die
or may continue as an independent diffusion.



\section{Questions:} 
\begin{enumerate}

    \item Do we get the same system using, say, a Wright--Fisher model?  (I assume so?)

    \item Can we prove convergence of the finite model to this one, under suitable assumptions?  A main difficulty here would be showing that the lineage spends all its time in the part of the range where $p(x)$ does not fluctuate significantly, i.e.\ is not too small\ldots but if this is not true, it would be even more interesting.

\end{enumerate}

\end{document}
