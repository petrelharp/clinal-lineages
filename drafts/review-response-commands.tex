\usepackage{lineno}
% \usepackage[hypertexnames=false]{hyperref}   % not working correctly
% \usepackage{latexml}

\linenumbers


%%%%%  PUT THIS IN HEADER OF FILE
% % Responses to reviews:
% \usepackage{lineno}
% \usepackage[hypertexnames=false]{hyperref}   % not working correctly
% \usepackage{latexml}

\linenumbers


%%%%%  PUT THIS IN HEADER OF FILE
% % Responses to reviews:
% \usepackage{lineno}
% \usepackage[hypertexnames=false]{hyperref}   % not working correctly
% \usepackage{latexml}

\linenumbers


%%%%%  PUT THIS IN HEADER OF FILE
% % Responses to reviews:
% \usepackage{lineno}
% \usepackage[hypertexnames=false]{hyperref}   % not working correctly
% \usepackage{latexml}

\linenumbers


%%%%%  PUT THIS IN HEADER OF FILE
% % Responses to reviews:
% \input{review-response-commands}
% % set this to show line numbers and include responses to reviews or not
% \newif\ifreviewresponses
% \reviewresponsestrue  % include them
% % \reviewresponsesfalse  % don't include them
% \newcommand{\responsefile}{pbio-reviews-19sept12-responses.tex}  % name of the review reponses file

% counters for reviewer points
%% instead do reviewer labels
% \newcounter{reviewer}
% \setcounter{reviewer}{0}
\newcommand{\thereviewer}{}
\newcounter{point}
\setcounter{point}{0}

% pass in to \reviewersection the label for this reviewer (i.e. \reviewersection{1} or \reviewersection{AE})
\newcommand{\reviewersection}[1]{\renewcommand{\thereviewer}{#1}
                  \setcounter{point}{0}
                  \section*{Reviewer \thereviewer:}}
% drawing from from http://tex.stackexchange.com/questions/2317/latex-style-or-macro-for-detailed-response-to-referee-report
%% arguments to \point are (name of the point, optional) and (content)
\newenvironment{point}[1]
        { \refstepcounter{point} \bigskip \hrule \medskip \noindent 
                \slshape {\fontseries{b}\selectfont (\thereviewer.\thepoint) #1} }
        { }
\newcommand{\reply}{\normalfont \medskip \noindent \textbf{Reply}:\ }   

% use this command in the text where a change addressing a reviewer point has occurred
% e.g. \revpoint{1}{3} for reviewer 1, point 3
\newcommand{\revpoint}[2]{\linelabel{rr:rev#1:#2}}
% and this one to refer to such a location, e.g. \revreffull{1}{3}
\newcommand{\revreffull}[2]{{(p.\ \pageref{rr:rev#1:#2}, l.\ \lineref{rr:rev#1:#2})}}
% but this version fills in reviewer and point automatically if called in the appropriate part of the reviews
\newcommand{\revref}{\revreffull{\thereviewer}{\thepoint}}

% or, this one to refer to a named linelabel
% e.g. if in the text there is a \linelabel{approx_eqn_point}
% refer to it with \llname{approx_eqn_point}
\newcommand{\llname}[1]{{(p.\ \pageref{#1}, l.\ \lineref{#1})}}

% put this where the reviews are to appear (at the end?)
\newcommand{\includereviews}{
    \ifreviewresponses
    \clearpage
    \setcounter{page}{1}
    \setcounter{section}{0}
    \setcounter{subsection}{0}
    \nolinenumbers
    % \begin{center}
    %   {\LARGE \bf Response to Reviews}
    % \end{center}
    \input{\responsefile}
    \fi
}

% Useful shortcuts
\newcommand{\rollover}{ \reply{The reviewer makes an excellent point that we have missed out entirely.  We have made all the changes suggested, down to the minutiae \revref.} }
\newcommand{\playdead}{ \reply{The reviewer makes an excellent point.  We have made an utterly trivial change {\revref} that we think deals entirely with the concern raised.} }
                                                                                                         
% from http://tex.stackexchange.com/questions/43648/why-doesnt-lineno-number-a-paragraph-when-it-is-followed-by-an-align-equation/55297#55297
\ifcsname{patchAmsMathEnvironmentForLineno}\endcsname
    \newcommand*\patchAmsMathEnvironmentForLineno[1]{%                                                       
      \expandafter\let\csname old#1\expandafter\endcsname\csname #1\endcsname                                
      \expandafter\let\csname oldend#1\expandafter\endcsname\csname end#1\endcsname                          
      \renewenvironment{#1}%                                                                                 
         {\linenomath\csname old#1\endcsname}%                                                               
         {\csname oldend#1\endcsname\endlinenomath}}%                                                        
    \newcommand*\patchBothAmsMathEnvironmentsForLineno[1]{%                                                  
      \patchAmsMathEnvironmentForLineno{#1}%                                                                 
      \patchAmsMathEnvironmentForLineno{#1*}}%                                                               
    \AtBeginDocument{%                                                                                       
    \patchBothAmsMathEnvironmentsForLineno{equation}%                                                        
    \patchBothAmsMathEnvironmentsForLineno{align}%                                                           
    \patchBothAmsMathEnvironmentsForLineno{flalign}%                                                         
    \patchBothAmsMathEnvironmentsForLineno{alignat}%                                                         
    \patchBothAmsMathEnvironmentsForLineno{gather}%                                                          
    \patchBothAmsMathEnvironmentsForLineno{multline}%                                                        
\fi

% % set this to show line numbers and include responses to reviews or not
% \newif\ifreviewresponses
% \reviewresponsestrue  % include them
% % \reviewresponsesfalse  % don't include them
% \newcommand{\responsefile}{pbio-reviews-19sept12-responses.tex}  % name of the review reponses file

% counters for reviewer points
%% instead do reviewer labels
% \newcounter{reviewer}
% \setcounter{reviewer}{0}
\newcommand{\thereviewer}{}
\newcounter{point}
\setcounter{point}{0}

% pass in to \reviewersection the label for this reviewer (i.e. \reviewersection{1} or \reviewersection{AE})
\newcommand{\reviewersection}[1]{\renewcommand{\thereviewer}{#1}
                  \setcounter{point}{0}
                  \section*{Reviewer \thereviewer:}}
% drawing from from http://tex.stackexchange.com/questions/2317/latex-style-or-macro-for-detailed-response-to-referee-report
%% arguments to \point are (name of the point, optional) and (content)
\newenvironment{point}[1]
        { \refstepcounter{point} \bigskip \hrule \medskip \noindent 
                \slshape {\fontseries{b}\selectfont (\thereviewer.\thepoint) #1} }
        { }
\newcommand{\reply}{\normalfont \medskip \noindent \textbf{Reply}:\ }   

% use this command in the text where a change addressing a reviewer point has occurred
% e.g. \revpoint{1}{3} for reviewer 1, point 3
\newcommand{\revpoint}[2]{\linelabel{rr:rev#1:#2}}
% and this one to refer to such a location, e.g. \revreffull{1}{3}
\newcommand{\revreffull}[2]{{(p.\ \pageref{rr:rev#1:#2}, l.\ \lineref{rr:rev#1:#2})}}
% but this version fills in reviewer and point automatically if called in the appropriate part of the reviews
\newcommand{\revref}{\revreffull{\thereviewer}{\thepoint}}

% or, this one to refer to a named linelabel
% e.g. if in the text there is a \linelabel{approx_eqn_point}
% refer to it with \llname{approx_eqn_point}
\newcommand{\llname}[1]{{(p.\ \pageref{#1}, l.\ \lineref{#1})}}

% put this where the reviews are to appear (at the end?)
\newcommand{\includereviews}{
    \ifreviewresponses
    \clearpage
    \setcounter{page}{1}
    \setcounter{section}{0}
    \setcounter{subsection}{0}
    \nolinenumbers
    % \begin{center}
    %   {\LARGE \bf Response to Reviews}
    % \end{center}
    \input{\responsefile}
    \fi
}

% Useful shortcuts
\newcommand{\rollover}{ \reply{The reviewer makes an excellent point that we have missed out entirely.  We have made all the changes suggested, down to the minutiae \revref.} }
\newcommand{\playdead}{ \reply{The reviewer makes an excellent point.  We have made an utterly trivial change {\revref} that we think deals entirely with the concern raised.} }
                                                                                                         
% from http://tex.stackexchange.com/questions/43648/why-doesnt-lineno-number-a-paragraph-when-it-is-followed-by-an-align-equation/55297#55297
\ifcsname{patchAmsMathEnvironmentForLineno}\endcsname
    \newcommand*\patchAmsMathEnvironmentForLineno[1]{%                                                       
      \expandafter\let\csname old#1\expandafter\endcsname\csname #1\endcsname                                
      \expandafter\let\csname oldend#1\expandafter\endcsname\csname end#1\endcsname                          
      \renewenvironment{#1}%                                                                                 
         {\linenomath\csname old#1\endcsname}%                                                               
         {\csname oldend#1\endcsname\endlinenomath}}%                                                        
    \newcommand*\patchBothAmsMathEnvironmentsForLineno[1]{%                                                  
      \patchAmsMathEnvironmentForLineno{#1}%                                                                 
      \patchAmsMathEnvironmentForLineno{#1*}}%                                                               
    \AtBeginDocument{%                                                                                       
    \patchBothAmsMathEnvironmentsForLineno{equation}%                                                        
    \patchBothAmsMathEnvironmentsForLineno{align}%                                                           
    \patchBothAmsMathEnvironmentsForLineno{flalign}%                                                         
    \patchBothAmsMathEnvironmentsForLineno{alignat}%                                                         
    \patchBothAmsMathEnvironmentsForLineno{gather}%                                                          
    \patchBothAmsMathEnvironmentsForLineno{multline}%                                                        
\fi

% % set this to show line numbers and include responses to reviews or not
% \newif\ifreviewresponses
% \reviewresponsestrue  % include them
% % \reviewresponsesfalse  % don't include them
% \newcommand{\responsefile}{pbio-reviews-19sept12-responses.tex}  % name of the review reponses file

% counters for reviewer points
%% instead do reviewer labels
% \newcounter{reviewer}
% \setcounter{reviewer}{0}
\newcommand{\thereviewer}{}
\newcounter{point}
\setcounter{point}{0}

% pass in to \reviewersection the label for this reviewer (i.e. \reviewersection{1} or \reviewersection{AE})
\newcommand{\reviewersection}[1]{\renewcommand{\thereviewer}{#1}
                  \setcounter{point}{0}
                  \section*{Reviewer \thereviewer:}}
% drawing from from http://tex.stackexchange.com/questions/2317/latex-style-or-macro-for-detailed-response-to-referee-report
%% arguments to \point are (name of the point, optional) and (content)
\newenvironment{point}[1]
        { \refstepcounter{point} \bigskip \hrule \medskip \noindent 
                \slshape {\fontseries{b}\selectfont (\thereviewer.\thepoint) #1} }
        { }
\newcommand{\reply}{\normalfont \medskip \noindent \textbf{Reply}:\ }   

% use this command in the text where a change addressing a reviewer point has occurred
% e.g. \revpoint{1}{3} for reviewer 1, point 3
\newcommand{\revpoint}[2]{\linelabel{rr:rev#1:#2}}
% and this one to refer to such a location, e.g. \revreffull{1}{3}
\newcommand{\revreffull}[2]{{(p.\ \pageref{rr:rev#1:#2}, l.\ \lineref{rr:rev#1:#2})}}
% but this version fills in reviewer and point automatically if called in the appropriate part of the reviews
\newcommand{\revref}{\revreffull{\thereviewer}{\thepoint}}

% or, this one to refer to a named linelabel
% e.g. if in the text there is a \linelabel{approx_eqn_point}
% refer to it with \llname{approx_eqn_point}
\newcommand{\llname}[1]{{(p.\ \pageref{#1}, l.\ \lineref{#1})}}

% put this where the reviews are to appear (at the end?)
\newcommand{\includereviews}{
    \ifreviewresponses
    \clearpage
    \setcounter{page}{1}
    \setcounter{section}{0}
    \setcounter{subsection}{0}
    \nolinenumbers
    % \begin{center}
    %   {\LARGE \bf Response to Reviews}
    % \end{center}
    \input{\responsefile}
    \fi
}

% Useful shortcuts
\newcommand{\rollover}{ \reply{The reviewer makes an excellent point that we have missed out entirely.  We have made all the changes suggested, down to the minutiae \revref.} }
\newcommand{\playdead}{ \reply{The reviewer makes an excellent point.  We have made an utterly trivial change {\revref} that we think deals entirely with the concern raised.} }
                                                                                                         
% from http://tex.stackexchange.com/questions/43648/why-doesnt-lineno-number-a-paragraph-when-it-is-followed-by-an-align-equation/55297#55297
\ifcsname{patchAmsMathEnvironmentForLineno}\endcsname
    \newcommand*\patchAmsMathEnvironmentForLineno[1]{%                                                       
      \expandafter\let\csname old#1\expandafter\endcsname\csname #1\endcsname                                
      \expandafter\let\csname oldend#1\expandafter\endcsname\csname end#1\endcsname                          
      \renewenvironment{#1}%                                                                                 
         {\linenomath\csname old#1\endcsname}%                                                               
         {\csname oldend#1\endcsname\endlinenomath}}%                                                        
    \newcommand*\patchBothAmsMathEnvironmentsForLineno[1]{%                                                  
      \patchAmsMathEnvironmentForLineno{#1}%                                                                 
      \patchAmsMathEnvironmentForLineno{#1*}}%                                                               
    \AtBeginDocument{%                                                                                       
    \patchBothAmsMathEnvironmentsForLineno{equation}%                                                        
    \patchBothAmsMathEnvironmentsForLineno{align}%                                                           
    \patchBothAmsMathEnvironmentsForLineno{flalign}%                                                         
    \patchBothAmsMathEnvironmentsForLineno{alignat}%                                                         
    \patchBothAmsMathEnvironmentsForLineno{gather}%                                                          
    \patchBothAmsMathEnvironmentsForLineno{multline}%                                                        
\fi

% % set this to show line numbers and include responses to reviews or not
% \newif\ifreviewresponses
% \reviewresponsestrue  % include them
% % \reviewresponsesfalse  % don't include them
% \newcommand{\responsefile}{pbio-reviews-19sept12-responses.tex}  % name of the review reponses file

% counters for reviewer points
%% instead do reviewer labels
% \newcounter{reviewer}
% \setcounter{reviewer}{0}
\newcommand{\thereviewer}{}
\newcounter{point}
\setcounter{point}{0}

% pass in to \reviewersection the label for this reviewer (i.e. \reviewersection{1} or \reviewersection{AE})
\newcommand{\reviewersection}[1]{\renewcommand{\thereviewer}{#1}
                  \setcounter{point}{0}
                  \section*{Reviewer \thereviewer:}}
% drawing from from http://tex.stackexchange.com/questions/2317/latex-style-or-macro-for-detailed-response-to-referee-report
%% arguments to \point are (name of the point, optional) and (content)
\newenvironment{point}[1]
        { \refstepcounter{point} \bigskip \hrule \medskip \noindent 
                \slshape {\fontseries{b}\selectfont (\thereviewer.\thepoint) #1} }
        { }
\newcommand{\reply}{\normalfont \medskip \noindent \textbf{Reply}:\ }   

% use this command in the text where a change addressing a reviewer point has occurred
% e.g. \revpoint{1}{3} for reviewer 1, point 3
\newcommand{\revpoint}[2]{\linelabel{rr:rev#1:#2}}
% and this one to refer to such a location, e.g. \revreffull{1}{3}
\newcommand{\revreffull}[2]{{(p.\ \pageref{rr:rev#1:#2}, l.\ \lineref{rr:rev#1:#2})}}
% but this version fills in reviewer and point automatically if called in the appropriate part of the reviews
\newcommand{\revref}{\revreffull{\thereviewer}{\thepoint}}

% or, this one to refer to a named linelabel
% e.g. if in the text there is a \linelabel{approx_eqn_point}
% refer to it with \llname{approx_eqn_point}
\newcommand{\llname}[1]{{(p.\ \pageref{#1}, l.\ \lineref{#1})}}

% put this where the reviews are to appear (at the end?)
\newcommand{\includereviews}{
    \ifreviewresponses
    \clearpage
    \setcounter{page}{1}
    \setcounter{section}{0}
    \setcounter{subsection}{0}
    \nolinenumbers
    % \begin{center}
    %   {\LARGE \bf Response to Reviews}
    % \end{center}
    \input{\responsefile}
    \fi
}

% Useful shortcuts
\newcommand{\rollover}{ \reply{The reviewer makes an excellent point that we have missed out entirely.  We have made all the changes suggested, down to the minutiae \revref.} }
\newcommand{\playdead}{ \reply{The reviewer makes an excellent point.  We have made an utterly trivial change {\revref} that we think deals entirely with the concern raised.} }
                                                                                                         
% from http://tex.stackexchange.com/questions/43648/why-doesnt-lineno-number-a-paragraph-when-it-is-followed-by-an-align-equation/55297#55297
\ifcsname{patchAmsMathEnvironmentForLineno}\endcsname
    \newcommand*\patchAmsMathEnvironmentForLineno[1]{%                                                       
      \expandafter\let\csname old#1\expandafter\endcsname\csname #1\endcsname                                
      \expandafter\let\csname oldend#1\expandafter\endcsname\csname end#1\endcsname                          
      \renewenvironment{#1}%                                                                                 
         {\linenomath\csname old#1\endcsname}%                                                               
         {\csname oldend#1\endcsname\endlinenomath}}%                                                        
    \newcommand*\patchBothAmsMathEnvironmentsForLineno[1]{%                                                  
      \patchAmsMathEnvironmentForLineno{#1}%                                                                 
      \patchAmsMathEnvironmentForLineno{#1*}}%                                                               
    \AtBeginDocument{%                                                                                       
    \patchBothAmsMathEnvironmentsForLineno{equation}%                                                        
    \patchBothAmsMathEnvironmentsForLineno{align}%                                                           
    \patchBothAmsMathEnvironmentsForLineno{flalign}%                                                         
    \patchBothAmsMathEnvironmentsForLineno{alignat}%                                                         
    \patchBothAmsMathEnvironmentsForLineno{gather}%                                                          
    \patchBothAmsMathEnvironmentsForLineno{multline}%                                                        
\fi
