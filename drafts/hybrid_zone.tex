
\documentclass[12pt]{amsart}
\usepackage{geometry} % see geometry.pdf on how to lay out the page. There's lots.
\geometry{a4paper} % or letter or a5paper or ... etc
% \geometry{landscape} % rotated page geometry

% See the ``Article customise'' template for come common customisations

\title{}
\author{}
\date{} % delete this line to display the current date

%%% BEGIN DOCUMENT
\begin{document}

\maketitle
%\tableofcontents

\section{Introduction}
-Studying hybrid zones is a great opportunity to understand the maintenance of distinct species, and the process of speciation. Here we consider hybrid zones that are a cline that is maintained by selection against hybrids (so-called tension zone). \\
-Gene flow across a hybrid zone may be impeded by selection against migrants. In older contact zones, loci that are associated with reduced fitness in heterozygotes will have steeper clines than the rest of the genome. There is probably a lot to learn from geographic patterns at loci neighboring selected loci (e.g. small block lengths indicate that there has been plenty of opportunity for recombination in heterozygous backgrounds)\\ 
-There has been a lot of investigation into the equilibrium state of hybrid zones, at both selected and linked loci. \\
-We consider the non-equilibrium state of hybrid zones, and patterns of linkage around selected loci. This will become increasingly relevant as studiers of HZs get genomic data and can reliably estimate block lengths. \\

	
\section{Methods}
\subsection{Model}
We consider a model in which two isolated subspecies, species $A$ and species $B$ come into contact at time $\tau$ generations in the past. Individuals move forward in time by sampling a displacement from a Gaussian distribution. In the simplest case of under-dominance, the fitness of an individual is determined by its genotype at a particular locus $L$, such that individuals that are heterozygous at locus $L$ have fitness reduced by $s>0$ compared to homozygotes.\\

We consider a lineage sampled in the present day, and determine its ancestry by its position at time $\tau$ generations in the past. The ancestry at locus $L$ is determined by its state, so that a lineage with identity $L_A$ is of ancestry $A$ and a lineage with identity $L_B$ is of ancestry $B$ regardless of its sampling location (we assume that the two species are fixed for alternate alleles at this locus). For every other locus, we keep track of $a$, the identity at $L$, across time and space. If at time $\tau$ a locus is on the same background as $L_A$, it is considered to be of ancestry $A$, and of ancestry $B$ otherwise. Recombination is modeled as a poisson process with rate $r$. The identity $a$ at locus $K$ changes when a recombination event occurs between $L$ and $K$ in an individual that is heterozygous at locus $L$. 
\\

As an approximation, we consider a cline of narrow width (look up cline width for underdominance, Barton/May?), such that the frequency of migrant alleles is exponential and negligible at distance away from the center of the hybrid zone. This means that recombination events that occur away from the center of the zone do not result in a change in $a$.\\


\subsection{}

\subsection{Simulations}
Forward in time simulations were used to asses theoretical predictions. We simulate discrete demes with nearest-neighbor migration and non-overlapping generations. Each generation, an individual moves to a neighboring deme with probability $m$. The direction in which it moves is determined by a coin-flip, except at edge demes, where an individual moves in one direction with probability $m/2$. Following migration, the next generation is produced by replacing each individual in a deme by randomly choosing two parents with probability weighted by the fitness of their genotypes. The genotype of the new individual is generated by simulating a recombined chromosome from each parent. This simulation keeps track of the ancestry at each location in the chromosome for all individuals. Each individual has 1 pair of chromosomes of length 1 Morgan. 

\end{document}
