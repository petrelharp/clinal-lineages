
\documentclass[12pt]{article}
\usepackage{geometry} % see geometry.pdf on how to lay out the page. There's lots.
\geometry{a4paper} % or letter or a5paper or ... etc
% \geometry{landscape} % rotated page geometry
\usepackage{natbib}
\usepackage{color}
% See the ``Article customise'' template for come common customisations
\newcommand{\alisa}[1]{{\em \color{red} #1}}

\title{Ancestry linkage in hybrid zones}
\author{}
\date{} % delete this line to display the current date

%%% BEGIN DOCUMENT
\begin{document}

\maketitle
%\tableofcontents

\section{Introduction}
The process of speciation commonly involves allopatric divergence in relative isolation \citep{Coyne2004}. When formerly isolated populations come back into contact, a gradient of ancestry can form across geography, resulting a hybrid zone \citep{Barton1985}. In addition to steep clines in ancestry, recently formed hybrid zones are characterized by a large physical scale of linkage disequilibrium in ancestry (ancestry-LD). If genetic variation can be exchanged between species via hybridization, both of these signals diminish over time throughout the genome; however, subsets of the genome that are under selection may be expected to maintain stable clines in ancestry as selection against migrant, or hybrid, genotypes impedes gene flow across the hybrid zone \citep{Barton1979a}. The influence of selection on linked loci also prevents the mixing of ancestry at regions surrounding these loci, the effect of which decreases with physical genomic distance \citep{Barton1986}.


Within a genome, the mixing of ancestries is brought about by the the process of recombination that takes place every generation. The expected distribution of lengths of unbroken ancestry have been previously described under neutral conditions \cite[e.g.][]{Gravel2012,Sedghifar2015}. Here we investigate expected genome-wide patterns of co-ancestry across a hybrid zone in the presence of selection. Selection against migrant ancestries in a hybrid zone results in reduced frequencies of foreign ancestry surrounding targets of selection. The effect on patterns of ancestral block lengths is, however, less well understood. Because of rapid removal from the population, we expect to observe few short blocks of foreign ancestry surrounding the selected locus, with this effect becoming more pronounced further away from the center of a hybrid zone.

Some words about how new genomic data will allow us to focus on tracts and not just individual loci. 

While population genetic models of hybrid zones have been extensively explored, these have typically focused on the equilibrium states of ancestry clines \cite{Barton1979a,Barton1986}. These models predict cline width to be narrowest at loci close to selected loci, and cline width at less tightly linked loci to be wider and more heavily influenced by hybrid zone age and rates of gene flow. As denser genotype data becomes more readily available, genome-wide patterns of linkage can be utilized in the analysis of gene hybrid zones. 

Genome-wide covariance in ancestry can provide information about the number of surviving recombination events that have taken place between ancestral genotypes and therefore provides information about the age of the hybrid zone, and the strength of selection at linked loci. Our goal here is  to understand how strength of selection, density of selected sites, recombination rate and hybrid zone age interact to produce observed patterns of ancestry in present day populations. This may be especially important for an understanding and interpretation of of non-equilibrium patterns of diversity in hybrid zones, as well as understanding the consequences of selection on the extent of mixing of ancestral genomes and, consequently, the maintenance of species distinctions. In the extreme case of zero fitness in first-generation hybrids, for example, paternal genomes will never mix and the stable hybrid zone will resemble a recently formed one. If hybrids have non-zero fitness, and some amount of introgression can occur, and patterns of introgression can provide information about the strength of selection, extent of hybridization, and timing of secondary contact. In this latter case, it may be of additional interest to establish whether the hybrid zone is in its equilibrium state. 

Something about how we just study the selection against hets model,
but we expect general conclusions to apply to other situations.

Expectations:
Conditional  on having identity B on the wrong side, far away from the center of the zone, the length of unbroken ancestry surrounding the selected locus should be relatively long. This is because an unfit haplotype is likely to be a recent migrant from the other side of the hybrid zone, and therefore will not have experienced many recombination events. 

%In the absence of selection, genome-wide covariance in ancestry can be used to estimate timing and gene-flow in hybrid zones \cite{Sedghifar2015}. 

%When formerly isolated populations come back into contact, a hybrid zone can form, and may be maintained stably by selection. In such instances, selection on some subset of the genome prevents mixing of genotypes between species. We refer to such hybrid zones as tension zones (as opposed to contact zones which reflect a non-equilibrium gradient in ancestry that will eventually be eroded by migration throughout the whole genome). Because the population genetic signatures around loci experiencing selection is going to differ from other regions of the genome, natural hybrid zones present exciting opportunities to understand species maintenance and the process of speciation.

%Gene-flow across a hybrid zone is impeded by selection against migrant, or hybrid, genotypes, and as a result, produces a cline in ancestry that is expected to be steepest at and nearby the selected locus. Elsewhere in the genome, clines in ancestry are wider, and are more heavily influenced by hybrid zone age and rates of migration. Historically, this has been the most relied-upon population genetic signature, and it is only recently that abundant genomic data has opened the possibility of studying patterns of linkage within hybrid zones. For example, patterns of linkage disequilibrium (LD) or tract length distribution across geographic space can provide information about zone age \cite{Sedghifar2015}. In tensions zones geographic patterns of linkage at loci neighboring targets of selection can potentially be a powerful source of information.

%Many of the traditional models of genetic variation in hybrid zones to date have focused on the equilibrium state of hybrid zones, at both selected and linked loci (lots of Barton). 

%-Studying hybrid zones is a great opportunity to understand the maintenance of distinct species, and the process of speciation. Here we consider hybrid zones that are a cline that is maintained by selection against hybrids (so-called tension zone). 
%-Gene flow across a hybrid zone may be impeded by selection against migrants. In older contact zones, loci that are associated with reduced fitness in heterozygotes will have steeper clines than the rest of the genome. There is probably a lot to learn from geographic patterns at loci neighboring selected loci (e.g. small block lengths indicate that there has been plenty of opportunity for recombination in heterozygous backgrounds)
%-There has been a lot of investigation into the equilibrium state of hybrid zones, at both selected and linked loci. 


	
\section{Methods}
\subsection{Model}
We consider a model in which two isolated subspecies, species $A$ and species $B$ come into contact at time $\tau$ generations in the past. Individuals move forward in time by sampling a displacement from a Gaussian distribution. In the simplest case of under-dominance, the fitness of an individual is determined by its genotype at a particular locus $L$, such that individuals that are heterozygous at locus $L$ have fitness reduced by $s>0$ compared to homozygotes.

To fix notation, we will say that species $A$ was on the ``left'' of the zone of contact,
which corresponds to spatial positions $x<0$.

We consider a lineage sampled in the present day, and determine its ancestry by its position at time $\tau$ generations in the past. The ancestry at locus $L$ is determined by its state, so that a lineage with identity $L_A$ is of ancestry $A$ and a lineage with identity $L_B$ is of ancestry $B$ regardless of its sampling location (we assume that the two species are fixed for alternate alleles at this locus). For every other locus, we keep track of $a$, the identity at $L$, across time and space. If at time $\tau$ a locus is on the same background as $L_A$, it is considered to be of ancestry $A$, and of ancestry $B$ otherwise. Recombination is modeled as a poisson process with rate $r$. The identity $a$ at locus $K$ changes when a recombination event occurs between $L$ and $K$ in an individual that is heterozygous at locus $L$. 


%As an approximation, we consider a cline of narrow width (look up cline width for underdominance, Barton/May?), such that the frequency of migrant alleles is exponential and negligible at distance away from the center of the hybrid zone. This means that recombination events that occur away from the center of the zone do not result in a change in $a$.


%%%%%%%% %%%%%%%%%%
\subsection{Analysis}

Consider a single locus, with alternate alleles fixed in the two parental populations,
for which heterozygotes have fitness $1-s$, and homozygotes are of equal fitness.
The deterministic theory predicts that after secondary contact,
these alleles will form a stable cline.
To study gene flow past this cline we first need to understand how the cline itself is formed.
Let $p(x,t)$ denote the frequency of the $A$ allele at spatial location $x$ and time $t$.  
Suppose that secondary contact occurred at time $t=0$, 
and that $p(x,0) = 1$ if $x<0$ and $p(x,0)=0$ if $x>0$.
(In discrete computations, a deme at $x=0$ has $p(0,0)=1/2$.)
Let $\sigma^2$ be the mean squared distance between parent and offspring.
Assuming local random assortment of alleles into diploids,
that the habitat and dispersal are homogeneous,
and that $s$ and $\sigma$ are small, 
the commonly-used equation that $p$ solves is
\begin{align} \label{eqn:cline_pde}
    \dot p = \frac{\sigma^2}{2} \Delta p + s p (1-p) (2p-1) ,
\end{align}
where $\dot p(t,x)$ is the time derivative and $\Delta p$ is the Laplacian,
as derived in \citep{bazykin}; for discussion of diploidy see \citep{diploidcline}.
There is not an exact analytic solution to this equation, 
but a good approximation to the equilibrium solution is that
$\lim_{t \to \infty} p(x,t) \approx (1+\tanh(-2x\sqrt{s}/\sigma))/2$ \citep{bazykin}.
Considering the numerous approximations going into this,
the important attributes are that 
the stable cline has width of order $\sigma/\sqrt{s}$,
and decays exponentially.
Furthermore, the cline is established through diffusion.
This is slowed down somewhat by selection against heterozygotes,
but is still diffusive, and so it will take time of order $1/s$
to reach its stable width.
(To confirm this, 
rescale space by $\sigma/s$ and time by $1/s$ in equation \eqref{eqn:cline_pde},
and see that the result doesn't depend on $\sigma$ or on $s$.)

Now suppose that the frequency profile of the selected allele $p(x,t)$ is given.
We will understand how lineages of sampled individuals behave,
both at and away from the selected site,
by following lineages as they move between backgrounds,
as done for instance in \citet{selectioncoal,ralph2015patchy}.
Consider the collection of $A$ alleles found at location $x$.
The expected number of such alleles produced by location $y$ is proportional to 
the number of $A$ alleles at $y$ multiplied by the probability of dispersing from $y$ to $x$;
looking in the other direction, this is also the probability that an $A$ allele at $x$ had a parent at $y$.
In other words, lineages move as the random walk determined by dispersal,
but biased towards areas that produce more of the selected allele that they carry.
Making the same assumptions underlying equation \eqref{eqn:cline_pde},
we show in Appendix XXX
that the lineage of an $A$ allele moves as Brownian motion with speed $\sigma$
with drift $s \grad \log p(x,t)$ (i.e., it is pushed uphill on $\log p$ at speed $s$).
Since $p=1$ far from the zone to the left, 
and is proportional to $\exp(-x\sqrt{s}/\sigma)$ to the right,
then roughly, lineages on ``their own'' side wander randomly,
while lineages on ``the wrong'' side are pushed at constant speed $\sigma/\sqrt{s}$ 
back towards the side where they are more common.
Since an $A$ allele must by definition have been inherited from the $A$ side of the barrier 
at the time of secondary contact,
this push must get more intense the closer it is to the time of secondary contact.
Lineages of $B$ alleles do the same thing, of course, but in the opposite direction.

The behavior of a lineage at a linked locus is similar,
except that when in heterozygotes the lineage may move between backgrounds:
if we follow back through time the lineage of a locus we find linked to an $A$ allele, 
it will first tend to be inherited from ancestors to the left (as $A$ lineages drift to the left),
but if it spends time in the clinal region where recombination may happen,
it may be found to have been inherited from a $B$-carrying individual,
whose ancestors will tend to be more from the right.
Suppose we sample an allele today at a locus at recombination distance $r$ to the selected site.
If $X_t$ is the location of its ancestor $t$ generations ago,
and $Z_t$ is the identity of the selected allele that ancestor carried,
then we say that $X$ moves as a diffusion pushed by either $\log(p)$ or $\log(1-p)$,
and that $Z$ jumps between $A$ and $B$ at rate either $r p$ or $r(1-p)$,
depending on the identity of $Z$.
In symbols,
\begin{align}
    \begin{aligned}
        dX_t &= \begin{cases}
            & s \grad \log(p(X_t,t)) dt + dB_t \qquad & \text{if } Z_t = A \\
            & s \grad \log(1-p(X_t,t)) dt + dB_t \qquad & \text{if } Z_t = B 
        \end{cases} \\
        \P\{ Z_{t+\epsilon} &= B \given Z_t = A \} = \epsilon r (1-p(X_t,t)) + O(\epsilon^2) \\
        \P\{ Z_{t+\epsilon} &= A \given Z_t = B \} = \epsilon r p(X_t,t) + O(\epsilon^2)  .
    \end{aligned}
\end{align}

\textbf{illustrative figure of a lineage moving back and forth?}

Clines at linked sites are given by
the probability that 
an allele sampled $t$ generations after secondary contact at location $x$,
at recombination distance $r$ to a selected allele of type $z$
is inherited from an individual of ancestry $A$,
where $z$ can be either $A$ or $B$.
Denote this probability $q_r(x,t,z)$.
In other words,
\begin{align}
    q_r(x,t,z) = \P^x \{Z_t = A\} .
\end{align}
The description above implies that $q_r$ solves the equation
\begin{align}
    \dot q_r(x,t,A) &= \grad \log(p(x,t)) \cdot \grad q_r(x,t,A) 
            + \frac{\sigma^2}{2} \Delta q_r(x,t,A)
            + r (1-p(x,t))(q_r(x,t,B)-q_r(x,t,A)) \\
    \dot q_r(x,t,B) &= \grad \log(1-p(x,t)) \cdot \grad q_r(x,t,B) 
            + \frac{\sigma^2}{2} \Delta q_r(x,t,B)
            + r (1-p(x,t))(q_r(x,t,A)-q_r(x,t,B))  ,
\end{align}
with boundary conditions $q_r(x,0,A)=1$ and $q_r(x,0,B)=0$.



%%%%%%%% %%%%%%%%%%
\subsection{Simulations}
Forward in time simulations were used to assess theoretical predictions. We simulate discrete demes with nearest-neighbor migration and non-overlapping generations. Each generation, the dispersal distance of an individual is sampled from a normal distribution, and then discretized to correspond to a deme that is a corresponding distance away. Following migration, the next generation is produced by replacing each individual in a deme by randomly choosing two parents with probability weighted by the fitness of their genotypes. The genotype of the new individual is generated by simulating a recombined chromosome from each parent. This simulation keeps track of the ancestry at each location in the chromosome for all individuals. Each individual has 2 pairs of chromosomes each of length 1 Morgan such that chromosome 1 contains a selected locus at position 0.5M, and chromosome 2 contains no selected loci.

Hybrid zones of ages 10, 100 and 1000 generations were simulated under conditions of $\sigma=1$ and $s=0.1$. From this we measure unbroken tract lengths to the right of the selected locus. These were compared to the output of neutral simulations with $s=0$


\subsection{Statistics}
With the above theory and simulations, we are able to make predictions about patterns surrounding selected loci. In reality, however, such loci are not known, so it is useful to have per-site statistics that may allow for detection of candidate targets of selection. The most straightforward measure is $l_I(b)$ the length of haplotype chunk that a sampled chunk containing genomic position $b$, conditioned on ancestry $I$ at the site.

Another statistic is the cumulative probability $C_I(b)$ of a sampled chunk encompassing position $b$ being longer than a chunk sampled from the distribution of all chunk lengths across the genome in the population. 

Additionally, we look at the mean value of these statistics in the two chunks, $b-$ and $b+$, that flank the one containing $b$.



%\subsection{Data}
%We compare the predictions of our model to genomic data obtained from individuals sampled a naturally occurring hybrid zone between \emph{Mus m. musculus/M. m. domesticus} \cite{Turner2011,Turner2014}. The genomic data generated by \citep{Turner2014} was for a mapping population generated by mating wild-caught individuals. Because some individuals were mated multiple times, the mapping population comprises many closely related individuals. We approximate wild population samples by restricting our analysis to genotyped individuals that had both parents sampled in the same location in such a way that none of the individuals shared a parent (see table for list of individuals). Approximately 280,000 SNPs were represented in the final dataset. 

%Ancestry blocks were estimated using ChromopainterV2 \cite{Lawson2012} and STRUCTUREv2 \cite{Falush2003} with $k=2$.


%Although we expect the age of this hybrid zone to correspond to historical human movement, and therefore be thousands of years old \cite{Teschke2008}, weighted the physical scale of ancestry-disequilibrium (both in the form of weighted ancestry-LD \citep{Loh2013} and ancestral block lengths) suggest relatively recent hybridization. Specifically, the high LD between quite distantly linked SNPs in the FS population suggests that a relatively high proportion of early-generation hybrids, whereas the rapid decay of LD in the the TU population reflects a much older hybridization event. Importance of geographic distribution of popgen patterns. 

%\begin{table}
%\begin{tabular}{l|c|c}
%Individual ID  & Population \citep{Turner2011} & Longitude(E)\\
%\citep{Turner2014}& \citep{Turner2011}&\\
%\hline
%FP112 & FS&11.665\\
%FP67 & FS &11.665\\
%FP114 & FS &11.665\\
%FP79 & FS &11.665\\
%FP152 &  FS &11.665\\
%FP11 & GL &11.965\\
%FP111&GL &11.965\\
%FP59 & HA  &11.721\\
%FP92 & HA &11.721\\
%FP7 &  HA  & 11.721\\
%FP45 & HA&11.721\\
%FP244& HO &11.693\\
%FP9&   RE&11.994\\
%FP5 &  RF&11.767\\
%FP78 &  SO &11.539\\
%FP52 &  SO &11.539\\
%FP180& SO &11.539\\
%FP14 & ST&11.539\\
%FP40 & ST&11.539\\
%FP145 & TU&11.748\\
%FP29&  TU&11.748\\
%FP135& TU&11.748\\
%\end{tabular}
%\end{table}

\section{Results}
\subsection{The distribution of ancestry tract lengths}
We are interested specifically in the distribution of the length of continuous tracts of ancestry surrounding the selected locus. Far away from the zone center variants under negative selection will be rare \citep[for demonstration of this theoretical result, see e.g.][]{May1975,Slatkin1973,Barton??}, however, when present an unfit migrant haplotype is expected to be longer than a neutral migrant haplotype. Figure X shows the estimated probability density (pdf), of unbroken tract of ancestry B surrounding the selected locus, conditioning on a haplotype being of ancestry B at the selected locus. Compared to the expected density of tract lengths surrounding neutral loci, tracts of ancestry B around the selected locus are longer.


\subsection{Number of ancestors}
Not sure what the result is here yet -- unexpected

\section{Discussion}
Topics include\\
- ignoring coalescent\\
- single locus model, selection is STRONG\\

Some words about how confidently calling tracts is the way of the future.\\
How can this be applied to detecting selection?

\bibliographystyle{molecularEcology}
\bibliography{library}

\end{document}
